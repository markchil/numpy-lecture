\documentclass[xcolor={x11names,table}]{beamer}

\usepackage[overlay,absolute]{textpos}
\setlength{\TPHorizModule}{1cm}
\setlength{\TPVertModule}{1cm}

\usepackage[version=4]{mhchem}

\mode<presentation>
{
\usetheme{default}
\useinnertheme{circles}
\setbeamercovered{transparent}
}

\setbeamertemplate{navigation symbols}{}

\setbeamertemplate{enumerate items}[default]

\setbeamertemplate{footline}{%
	\hfill\scriptsize\textsf{\insertframenumber{}/\inserttotalframenumber}\hspace{0.1cm}\vspace{0.1cm}%
}

\usecolortheme{beaver}
\setbeamercolor{enumerate item}{fg=darkred}
\setbeamercolor{enumerate subitem}{fg=darkred}
\setbeamercolor{enumerate subsubitem}{fg=darkred}
\setbeamercolor{itemize item}{fg=darkred}
\setbeamercolor{itemize subitem}{fg=darkred}
\setbeamercolor{itemize subsubitem}{fg=darkred}
\setbeamercolor{section number projected}{bg=darkred}

\usepackage{graphicx}
\usepackage{url}
\urlstyle{same}

\usepackage[separate-uncertainty=true,detect-none,mode=text]{siunitx}

\usepackage{listings}
%\usepackage{lstlinebgrd}
\lstset{numbersep=4pt, %
	language=Python, % {[ANSI]C}, {[ISO]C++}, {[90]Fortran}, Python, IDL, Java, bash, sh
	numbers=left, % set to none to get rid of line numbers by default
	basicstyle=\ttfamily\small, % \footnotesize
	%keywordstyle=\color{blue}, %
	%identifierstyle=\color{RoyalBlue4}, %
	commentstyle=\color{Green4}, %
	stringstyle=\color{violet}, %
	showstringspaces=false, % set to true to show spaces visually
	numberstyle=\ttfamily\tiny\color{red}, %
	breaklines=true, %
	tabsize=4, %
	inputpath=../src/cell_files
}

\usepackage{tikz}
\usetikzlibrary{positioning}
\usetikzlibrary{calc}
\usetikzlibrary{fit}
\tikzstyle{arrow} = [thick,->,>=stealth]

\usepackage{booktabs}

\newcommand{\stabletitle}{Scientific Computing in Python:\\A NumPy Crash-Course Cluedump}

\title{\stabletitle}

\author[markchil] 
{Mark Chilenski\\(markchil)}

\date{January 25, 2022}

\usepackage{ragged2e}

\newcommand{\diff}{\ensuremath{\mathsf{d}}}
\newcommand{\eu}[1][]{\ensuremath{\mathsf{e}^{#1}}}
\newcommand{\pd}[2]{\frac{\partial #1}{\partial #2}}
\newcommand{\TT}{\ensuremath{\mathsf{T}}}

\newcommand{\vect}[1]{\mathbfsf{#1}}
\newcommand{\tens}[1]{\mathbf{#1}}

\DeclareMathOperator{\corr}{corr}
\DeclareMathOperator{\cov}{cov}
\DeclareMathOperator*{\argmax}{arg\,max}
\DeclareMathOperator*{\argmin}{arg\,min}

\newcommand{\todo}[1]{\textcolor{red}{\large\bfseries #1}}

\graphicspath{{../graphics/}}

\AtBeginSection[]
{
  \begin{frame}<beamer>
    \frametitle{\stabletitle}
    \tableofcontents[currentsection]
  \end{frame}
}

\begin{document}

\begin{frame}
 	\titlepage
	
	\begin{center}
		\item You can download the code to follow along at: \url{https://github.com/markchil/numpy-lecture}
	\end{center}
\end{frame}

\begin{frame}
	\frametitle{Intro}
	Goals for today:
	\begin{itemize}
		\item Introduce the core elements of NumPy
		\item Show how to write clean, fast code using NumPy
	\end{itemize}
	Prerequisites:
	\begin{itemize}
		\item Basic familiarity with Python
	\end{itemize}
	You can download the code to follow along at: \url{https://github.com/markchil/numpy-lecture}
\end{frame}

\begin{frame}
	\frametitle{\stabletitle}
	\tableofcontents
\end{frame}

\section{Background and Motivating Example}

\begin{frame}
	\frametitle{What Is NumPy?}
	\begin{itemize}
		\item NumPy is the main Python package for performing numerical computations
		\item NumPy provides data structures which enable efficient manipulation of arrays of data with expressive syntax
	\end{itemize}
\end{frame}

\begin{frame}
	\frametitle{Motivating Example:\\Evaluating $z=ax^{2}+by^{2}+cxy$ Over a 2d Grid}
	Without NumPy, setting up the $x$ and $y$ grids is pretty ugly:
	\lstinputlisting[firstline=4,lastline=5]{examples_annotated_3.py}
	
	Evaluating the function itself requires nested for-loops:
	\lstinputlisting[firstline=4,firstnumber=3]{examples_annotated_4.py}
\end{frame}

\begin{frame}
	\frametitle{Evaluating $z=ax^{2}+by^{2}+cxy$ Over a 2d Grid With NumPy}
	First, we import NumPy:
	\lstinputlisting[firstline=4,lastline=4]{examples_annotated_1.py}
	
	Defining our grids is easy with \texttt{np.linspace}:
	\lstinputlisting[firstline=4,lastline=6,firstnumber=2]{examples_annotated_6.py}
	
	And evaluating the function itself requires no loops:
	\lstinputlisting[firstline=4,firstnumber=4]{examples_annotated_7.py}
\end{frame}

\begin{frame}
	\frametitle{Performance Comparison}
	\begin{center}
		\begin{tabular}{rl}
			Pure Python: & \SI{14.2}{ms}\\
			NumPy: & \SI{138}{\micro s}\\
		\end{tabular}
	\end{center}
	\begin{itemize}
		\item The NumPy solution is not only cleaner (no nested for-loops), it is \emph{100$\times$ faster}
		\item NumPy obtains this speed by performing the majority of its operations in fast, compiled C/Fortran code: you write your code in Python, but NumPy does the heavy lifting in compiled libraries
	\end{itemize}
\end{frame}

\section{Storing $N$-Dimensional Data: The \texttt{ndarray} Class}

\begin{frame}
	\frametitle{The \texttt{ndarray} Class}
	\begin{itemize}
		\item The \texttt{ndarray} class stores arrays of $N$-dimensional data
		\begin{itemize}
			\item Usually just called ``array'' or ``numpy array''
		\end{itemize}
		\item A given array \texttt{x} consists of:
		\begin{itemize}
			\item A chunk of memory
			\item A datatype which indicates what type of value is stored in the memory (\texttt{x.dtype})
			\item A shape which indicates how the elements are arranged in an $N$-dimensional grid (\texttt{x.shape})
		\end{itemize}
	\end{itemize}
\end{frame}

\section{Creating Arrays}

\begin{frame}
	\frametitle{Creating Arrays From Pure Python Data Structures}
	\begin{itemize}
		\item Create an array from an existing list:
		\lstinputlisting[firstline=4,lastline=7]{examples_annotated_9.py}
		\begin{itemize}
			\item NumPy tries to guess the \texttt{dtype} based on the input: need to specify \texttt{dtype=float} here to avoid defaulting to \texttt{int} based on the input
		\end{itemize}
		\item Create a 2d array using nested lists:
		\lstinputlisting[firstline=4,lastline=9]{examples_annotated_10.py}
	\end{itemize}
\end{frame}

\begin{frame}
	\frametitle{Creating Arrays With Specified Shapes}
	\begin{itemize}
		\item Create an array of zeros:
		\lstinputlisting[firstline=4,lastline=10]{examples_annotated_11.py}
		\item Create an array of ones:
		\lstinputlisting[firstline=4,lastline=6]{examples_annotated_12.py}
	\end{itemize}
\end{frame}

%\begin{frame}
%	\frametitle{Creating an Array From Another Array}
%	\begin{itemize}
%		\item Create an array with the same shape and dtype as another array:
%		\lstinputlisting[firstline=4,lastline=12]{examples_annotated_13.py}
%		\item You can also override the dtype:
%		\lstinputlisting[firstline=4,lastline=6]{examples_annotated_14.py}
%	\end{itemize}
%\end{frame}

\begin{frame}
	\frametitle{Creating Grids of Numbers}
	\begin{itemize}
		\item We already saw how to create linearly-spaced numbers:
		\lstinputlisting[firstline=4,lastline=6]{examples_annotated_15.py}
		\item We can also create logarithmically-spaced numbers:
		\lstinputlisting[firstline=4,lastline=6]{examples_annotated_16.py}
		\begin{itemize}
			\item You specify the \emph{exponent} of the starting and ending points
			\item The default base is 10, you can change this with the \texttt{base} keyword
		\end{itemize}
	\end{itemize}
\end{frame}

\section{Indexing: Getting Elements Into (and out of) Arrays}

\begin{frame}
	\frametitle{Indexing Into 1d Arrays}
	You can index into 1d arrays just like with Python lists:
	\lstinputlisting[firstline=4,lastline=6]{examples_annotated_18.py}
	\begin{itemize}
		\item Indices start from zero:
		\lstinputlisting[firstline=4,lastline=5]{examples_annotated_19.py}
		\item Negative indexing goes from the end of the array, with $-1$ being the last element:
		\lstinputlisting[firstline=4,lastline=5]{examples_annotated_20.py}
	\end{itemize}
\end{frame}

\begin{frame}
	\frametitle{Slice Indexing}
	\lstinputlisting[firstline=4,lastline=6]{examples_annotated_18.py}
	You can access a range of elements using a slice:
	\lstinputlisting[firstline=4,lastline=5]{examples_annotated_21.py}
	\begin{itemize}
		\item The number before the colon is the element the slice starts from
		\item The number after the colon is \emph{one more than} the element the slice stops at
	\end{itemize}
\end{frame}

\begin{frame}
	\frametitle{Slice Indexing}
	\lstinputlisting[firstline=4,lastline=6]{examples_annotated_18.py}
	\begin{itemize}
		\item You can specify a stride for the slice:
		\lstinputlisting[firstline=4,lastline=5]{examples_annotated_22.py}
		\item As a shorthand, you can omit the start index (it will default to \texttt{0}) and/or the stop index (it will default to \texttt{len(x)}):
		\lstinputlisting[firstline=4,lastline=7]{examples_annotated_23.py}
	\end{itemize}
\end{frame}

\begin{frame}
	\frametitle{Exacting Elements at Specific Indices}
	\lstinputlisting[firstline=4,lastline=6]{examples_annotated_18.py}
	You can use an array (or list) of integers to extract the elements at specific values:
	\lstinputlisting[firstline=4,lastline=5]{examples_annotated_24.py}
\end{frame}

\begin{frame}
	\frametitle{Boolean Indexing}
	\lstinputlisting[firstline=4,lastline=6]{examples_annotated_18.py}
	\begin{itemize}
		\item You can use an array (or list) of booleans with the same length as the array to extract specific elements:
		\lstinputlisting[firstline=4,lastline=5]{examples_annotated_25.py}
		\item The use of boolean array expressions for indexing is extremely powerful:
		\lstinputlisting[firstline=4,lastline=5]{examples_annotated_26.py}
	\end{itemize}
\end{frame}

\begin{frame}
	\frametitle{Indexing Into 2d Arrays}
	You can provide a comma-separated list of any of the types of indexing shown above to extract elements from a 2-dimensional array:
	\lstinputlisting[firstline=4,lastline=10]{examples_annotated_27.py}
	The first index is the row index, the second index is the column index:
	\lstinputlisting[firstline=4,lastline=5]{examples_annotated_28.py}
\end{frame}

\begin{frame}
	\frametitle{Indexing Into 2d Arrays}
	\lstinputlisting[firstline=4,lastline=10]{examples_annotated_27.py}
	To get all of the entries along a given axis (e.g., an entire row), use the slice operator \texttt{:} with no start, end, or stride indicated:
	\lstinputlisting[firstline=4,lastline=5]{examples_annotated_29.py}
	You can omit the trailing \texttt{:} when accessing entire rows:
	\lstinputlisting[firstline=4,lastline=5]{examples_annotated_30.py}
\end{frame}

\begin{frame}
	\frametitle{Indexing Into Higher-Dimensional Arrays}
	Pretty much the same as for 2d: provide a comma-separated list of indices, one for each dimension
	\lstinputlisting[firstline=4,lastline=4]{examples_annotated_33.py}
	You can omit any number of trailing \texttt{:}'s:
	\lstinputlisting[firstline=4,lastline=15]{examples_annotated_34.py}
\end{frame}

\begin{frame}
	\frametitle{Indexing Into Higher-Dimensional Arrays}
	\lstinputlisting[firstline=4,lastline=4]{examples_annotated_33.py}
	You can omit any number of leading \texttt{:}'s by using \texttt{...}:
	\lstinputlisting[firstline=4,lastline=13]{examples_annotated_35.py}
\end{frame}

\begin{frame}
	\frametitle{Adding New Dimensions With Indexing}
	You can add a new dimension (with size 1) to an array by including \texttt{np.newaxis} in the index expression:
	\lstinputlisting[firstline=4,lastline=9]{examples_annotated_36.py}
	
	This will be very useful when setting up certain operations
\end{frame}

\section{Array Operations}

\begin{frame}
	\frametitle{Vectorization: The Key to Speed}
	\begin{itemize}
		\item Almost all NumPy operations are \emph{vectorized}: a single function call causes the same operation to be applied on every element of an array
		\item This avoids the need to write loops in Python: instead, you let NumPy do the loop in its fast, compiled libraries
		\item Code which avoids loops will (almost) always be faster than code with loops, and is usually cleaner/easier to understand
	\end{itemize}
\end{frame}

\begin{frame}
	\frametitle{Speed Test: Exponentiation}
	Let's try squaring a bunch of numbers:
	\lstinputlisting[firstline=4,lastline=4]{examples_annotated_37.py}
	First we do it with a loop:
	\lstinputlisting[firstline=4,lastline=8]{examples_annotated_38.py}
	Then with a single line using NumPy's vectorized exponentiation operation:
	\lstinputlisting[firstline=4,lastline=5]{examples_annotated_40.py}
	\texttt{square\_loop()} takes \SI{3}{ms}, \texttt{square\_numpy()} takes \SI{5.0}{\micro s}: \emph{the vectorized version is 600$\times$ faster}
\end{frame}

\begin{frame}
	\frametitle{Unary Operations}
	Unary operations are applied to each element:
	\lstinputlisting[firstline=4,lastline=6]{examples_annotated_42.py}
	Scalar addition:
	\lstinputlisting[firstline=4,lastline=6]{examples_annotated_43.py}
	Scalar multiplication:
	\lstinputlisting[firstline=4,lastline=6]{examples_annotated_44.py}
	Scalar division:
	\lstinputlisting[firstline=4,lastline=6]{examples_annotated_45.py}
\end{frame}

\begin{frame}
	\frametitle{Unary Operations}
	\lstinputlisting[firstline=4,lastline=6]{examples_annotated_42.py}
	Negation:
	\lstinputlisting[firstline=4,lastline=6]{examples_annotated_46.py}
	Exponentiation:
	\lstinputlisting[firstline=4,lastline=6]{examples_annotated_47.py}
\end{frame}

\begin{frame}
	\frametitle{Unary Operations}
	Some operations are provided as functions instead of operators:
	\lstinputlisting[firstline=4,lastline=6]{examples_annotated_42.py}
	Square root:
	\lstinputlisting[firstline=4,lastline=6]{examples_annotated_48.py}
	Exponential ($y=e^{x}$):
	\lstinputlisting[firstline=4,lastline=6]{examples_annotated_49.py}
\end{frame}

\begin{frame}
	\frametitle{Unary Operations}
	Some operations are provided as functions instead of operators:
	\lstinputlisting[firstline=4,lastline=6]{examples_annotated_42.py}
	(Natural) logarithm (note the warning from taking $\log(0)$):
	\lstinputlisting[firstline=4,lastline=7]{examples_annotated_50.py}
	(Trigonometric) sine:
	\lstinputlisting[firstline=4,lastline=6]{examples_annotated_51.py}
\end{frame}

\begin{frame}
	\frametitle{Binary Operations}
	\begin{itemize}
		\item Binary operations act element-by-element
		\item In their most basic form, this means that the two operands must have the same shape
		\begin{itemize}
			\item But we'll see an exception to this when we get to broadcasting
		\end{itemize}
	\end{itemize}
\end{frame}

\begin{frame}
	\frametitle{Examples of Binary Operations}
	\lstinputlisting[firstline=4,lastline=9]{examples_annotated_52.py}
	Addition:
	\lstinputlisting[firstline=4,lastline=6]{examples_annotated_53.py}
	Division (note warning on divide-by-zero):
	\lstinputlisting[firstline=4,lastline=7]{examples_annotated_55.py}
\end{frame}

\begin{frame}
	\frametitle{Broadcasting: Binary Operations on Differently-Sized Arrays}
	\begin{itemize}
		\item Broadcasting is a powerful syntax for efficiently performing calculations on ``tensor product'' grids of inputs
		\item It is a very useful tool for removing the need to use loops (and hence dramatically speeding up your code)
	\end{itemize}
\end{frame}

\begin{frame}
	\frametitle{Basic Broadcasting Example: $z=xy$}
	Let's try to compute $z=xy$ over the grid $-1\leq x\leq 0$, $0\leq y\leq 1$:
	\lstinputlisting[firstline=4,lastline=5]{examples_annotated_57.py}
	We can't directly multiply \texttt{x} and \texttt{y} because they have different shapes:
	\lstinputlisting[firstline=6,lastline=6]{examples_annotated_58.py}
	\lstinputlisting[firstline=11,lastline=14]{examples_annotated_58.py}
\end{frame}

\begin{frame}
	\frametitle{Basic Broadcasting Example: $z=xy$}
	\lstinputlisting[firstline=4,lastline=5]{examples_annotated_57.py}
	We can use \texttt{np.newaxis} to reshape \texttt{x} and \texttt{y}:
	\lstinputlisting[firstline=4,lastline=9]{examples_annotated_59.py}
	And this allows us to compute the desired expression:
	\lstinputlisting[firstline=4,lastline=6]{examples_annotated_60.py}
\end{frame}

\begin{frame}
	\frametitle{What Just Happened? Full Details of Broadcasting}
	Two arrays can be \emph{broadcast} together if their shapes are \emph{compatible} according to the following rules:
	\begin{enumerate}
		\item If one array has fewer dimensions than the other, add 1's at the \emph{beginning} of the lower-dimensional array's shape
		\item For each element of the shapes, either:
		\begin{itemize}
			\item The two arrays must have the same number of elements along that dimension, or
			\item One (or both) of the arrays must have 1 element along that dimension
		\end{itemize}
	\end{enumerate}
	Whenever there is a 1 in one array's shape, NumPy pretends that the array has been repeated along that axis however many times are necessary to match the shape of the other array
\end{frame}

\begin{frame}
	\frametitle{Example of Broadcasting}
	Whenever there is a 1 in one array's shape, NumPy pretends that the array has been repeated along that axis however many times are necessary to match the shape of the other array:
	\lstinputlisting[firstline=4,lastline=12]{examples_annotated_61.py}
\end{frame}

\begin{frame}
	\frametitle{Aggregating Information Along Array Axes}
	\lstinputlisting[firstline=4,lastline=8]{examples_annotated_62.py}
	
	Sum of elements:
	\lstinputlisting[firstline=4,lastline=5]{examples_annotated_63.py}
	
	Minimum value:
	\lstinputlisting[firstline=4,lastline=5]{examples_annotated_64.py}
	
	Maximum value:
	\lstinputlisting[firstline=4,lastline=5]{examples_annotated_65.py}
	
	Mean value:
	\lstinputlisting[firstline=4,lastline=5]{examples_annotated_66.py}
\end{frame}

\begin{frame}
	\frametitle{Aggregating in Multidimensional Arrays}
	On multidimensional arrays, the default behavior is to aggregate across all of the elements of the array to yield a scalar:
	\lstinputlisting[firstline=4,lastline=6]{examples_annotated_68.py}
\end{frame}

\begin{frame}
	\frametitle{Aggregating in Multidimensional Arrays}
	\lstinputlisting[firstline=4,lastline=4]{examples_annotated_68.py}
	You can specify an axis to do the aggregation on -- the result will have one fewer dimension than the input:
	\lstinputlisting[firstline=4,lastline=8]{examples_annotated_69.py}
\end{frame}

\begin{frame}
	\frametitle{Aggregating in Multidimensional Arrays}
	\lstinputlisting[firstline=4,lastline=4]{examples_annotated_68.py}
	You can specify also aggregate over multiple dimensions at once:
	\lstinputlisting[firstline=4,lastline=8]{examples_annotated_70.py}
\end{frame}

\begin{frame}
	\frametitle{Aggregating Without Removing Dimensions}
	Suppose we want to normalize each row of a matrix by its 2-norm:
	\lstinputlisting[firstline=4,lastline=9]{examples_annotated_71.py}
	\texttt{x} and \texttt{x\_norm} are not compatible: we can't just take \texttt{x/x\_norm}
\end{frame}

\begin{frame}
	\frametitle{Aggregating Without Removing Dimensions}
	Use the \texttt{keepdims} flag to ensure the resulting shapes are compatible:
	\lstinputlisting[firstline=4,lastline=9]{examples_annotated_72.py}
\end{frame}

\section{The Scientific Python Ecosystem}

\begin{frame}
	\frametitle{Doing Stuff With Arrays: The Scientific Python Ecosystem}
	Many packages use NumPy to move data around and do stuff, here are a few:
	\begin{itemize}
		\item NumPy (\url{numpy.org}) itself has subpackages for:
		\begin{itemize}
			\item Linear algebra (\texttt{numpy.linalg})
			\item Taking discrete Fourier transforms (\texttt{numpy.fft})
			\item Generating random numbers (\texttt{numpy.random})
		\end{itemize}
		\item SciPy (\url{scipy.org}) was developed jointly with NumPy, and provides many high-level routines, including:
		\begin{itemize}
			\item Numeric integration (including ODE solvers) (\texttt{scipy.integrate})
			\item Optimization (\texttt{scipy.optimize})
			\item Interpolation/smoothing (\texttt{scipy.interpolate})
			\item Statistics (\texttt{scipy.stats})
		\end{itemize}
		\item Matplotlib (\url{matplotlib.org}) provides a variety of plotting routines
		\item scikit-learn (\url{scikit-learn.org}) provides a huge range of machine learning algorithms and supporting tools
	\end{itemize}
\end{frame}

\end{document}